%!TEX TS-program = xelatex
%!TEX encoding = UTF-8

% LaTeX source for the errata of the book ``代数学方法'' in Chinese
% Copyright 2025  李文威 (Wen-Wei Li).
% Permission is granted to copy, distribute and/or modify this
% document under the terms of the Creative Commons
% Attribution 4.0 International (CC BY 4.0)
% http://creativecommons.org/licenses/by/4.0/

% 《代数学方法》卷一勘误表 / 李文威
% 使用自定义的文档类 AJerrata.cls. 自动载入 xeCJK.

\documentclass{AJerrata}

\usepackage{unicode-math}

\usepackage[unicode, colorlinks, psdextra, bookmarksnumbered,
	pdfpagelabels=true,
	pdfauthor={李文威 (Wen-Wei Li)},
	pdftitle={代数学方法卷一勘误},
	pdfkeywords={}
]{hyperref}

\setmainfont[
	BoldFont={texgyretermes-bold.otf},
	ItalicFont={texgyretermes-italic.otf},
	BoldItalicFont={texgyretermes-bolditalic.otf},
	PunctuationSpace=2
]{texgyretermes-regular.otf}

\setsansfont[
	BoldFont=FiraSans-Bold.otf,
	ItalicFont=FiraSans-Italic.otf
]{FiraSans-Regular.otf}

\setCJKmainfont[
	BoldFont=Noto Serif CJK SC Bold
]{Noto Serif CJK SC}

\setCJKsansfont[
	BoldFont=Noto Sans CJK SC Bold
]{Noto Sans CJK SC}

\setCJKfamilyfont{emfont}[
	BoldFont=FandolHei-Regular.otf
]{FandolHei-Regular.otf}	% 强调用的字体

\renewcommand{\em}{\bfseries\CJKfamily{emfont}} % 强调

\setmathfont[
	Extension = .otf,
	math-style= TeX,
]{texgyretermes-math}

\usepackage{mathrsfs}
\usepackage{stmaryrd} \SetSymbolFont{stmry}{bold}{U}{stmry}{m}{n}	% 避免警告 (stmryd 不含粗体故)
% \usepackage{array}
% \usepackage{tikz-cd}  % 使用 TikZ 绘图
\usetikzlibrary{positioning, patterns, calc, matrix, shapes.arrows, shapes.symbols}

\usepackage{myarrows}				% 使用自定义的可伸缩箭头
\usepackage{mycommand}				% 引入自定义的惯用的命令


\title{\bfseries 代数学方法(第一卷)勘误表 \\ 跨度: 2025 迄今 }
\author{李文威}
\date{\today}

\begin{document}
	\maketitle
	以下页码等信息参照高等教育出版社 2025 年修订之《代数学方法》第一卷, ISBN: 978-7-04-050725-6. 这些错误将在下一批重印的版本改正.

	\begin{Errata}
		\item[命题 2.6.9 的证明第二行]
		\Orig $= \Hom_{\mathcal{C}_2}(\cdot, GY)$
		\Corr $= \Hom_{\mathcal{C}_1}(\cdot, GY)$
		\Thx{感谢王炜乔指正}
		
		\item[第五章习题 11]
		\Orig $Z(P,1) = \mu(\hat{0},\hat{1}) = \sum_i (-1)^i c_i$
		\Corr $Z(P,1) = \mu(\hat{0},\hat{1}) + 1 = \sum_{i \geq 2} (-1)^i c_i$
		
		\item[命题 6.8.6 的证明第二行]
		\Orig $x-x' \in \Ker$
		\Corr $x - (x'\,\text{的像}) \in \Ker$
		\Thx{感谢赵新雨指正}
		
		\item[第六章习题 5]
		\Orig ...作为子理想插入...
		\Corr ...作为``因子''插入...
%		\Thx{}

		\item[\S 7.4 倒数第二个图表]
		将图表底部的水平箭头下方的 $c(A, B)$ 改为 $c_\epsilon(A, B)$.
		
		\item[第七章习题 6 第五行]
		\Orig ..., 其中 $v$ 仅是符号, ...
		\Corr ..., 其中 $v$ 仅是符号, $\lambda \in F^{\times}$, ...
		
		\item[第七章习题 12]
		\Orig $w_i = \sum_{i=1}^m a_{ij} v_j$
		\Corr $w_i = \sum_{j=1}^m a_{ij} v_j$
		
		\item[第八章习题 7]
		\Orig $\Aut_F(F(X))$
		\Corr $\Aut_F(F(X))^{\mathrm{op}}$
		
		\Orig $\bigl( \begin{smallmatrix} a & b \\ c & d \end{smallmatrix} \bigr) \cdot X$
		\Corr $X \cdot \bigl( \begin{smallmatrix} a & b \\ c & d \end{smallmatrix} \bigr)$ ...
		
		\item[第九章习题 3 (i)]
		\Orig $\left( 1 - f_i(x) \right)^{q-1}$
		\Corr $\left( 1 - f_i(x)^{q-1} \right)$
	\end{Errata}
\end{document}
